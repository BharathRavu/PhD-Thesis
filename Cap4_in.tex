

\chapter{HJB equation and viscosity solution}\label{Chapter4}
%\blindtext
\minitoc% Creating an actual minitoc

\vspace{5em}

In this chapter we present a brief introduction to the use of the \textbf{dynamic programming principle} for solving stochastic control problems. 
The fundamental equation of dynamic programming is a nonlinear evolution equation for the \textbf{value function}.
The value function is the optimum value of the payoff considered as a function of the initial data.
This principle was introduced in the 1950s by Bellman, see \cite{Bellman}. 
For controlled Lévy processes,
the approach yields a certain nonlinear PIDE, usually of first or second order, called Hamilton-Jacobi-Bellman (HJB) equation. 
Under appropriate conditions the value function satisfies the HJB equation.

The theory of viscosity solutions, provides a convenient framework in which to study HJB equations.
Typically, the value function is not smooth enough to satisfy the HJB in the classical sense. However, under certain assumptions, it is the unique viscosity
solution of the HJB equation with appropriate boundary conditions. For a general review of the theory of viscosity solution see \cite{CIL92}.

The first notion of viscosity solution has been introduced by \cite{CL83} for first order partial differential equations. The theory has
been immediately extended by \cite{PLL83} for second order PDEs.
Uniqueness results for general second order equations are presented in \cite{Je88}, \cite{Is89} and \cite{IsLi90}. 
The main development of these works is the \emph{Ishii's lemma}, which plays a key role in most of the uniqueness proofs.
The viscosity solution theory has been further extended by \cite{Soner86b}, \cite{Soner86} and \cite{Sayah91} for piecewise deterministic processes with random jumps.
An important paper for viscosity solution of Lévy-type PIDEs (PIDEs involving the generator of a Lévy process) is \cite{BaIm08}, where the Ishii's
lemma is extended. 
Other important works that deserve to be mentioned are \cite{Ph98}, that analyzes HJB equations for optimal stopping problems under Lévy processes and \cite{CoVo05} that analyze
the solution of linear PIDEs derived from option prices problems (plain vanilla and barrier options) for finite and infinite activity Lévy processes.
\vspace{1em}

\section{Optimal control framework}\label{Optimal_control_framework}

We consider a framework where the state of the system is governed by the controlled SDE:
\begin{align}\label{controlled_SDE}
 dX_t =& b(t,X_{t-},\alpha_t) dt + \sigma (t,X_{t-},\alpha_t) dW_t \\ \nonumber
      &+ \int_{\R} \gamma(t,X_{t-},\alpha_t,z) \tilde N(dt,dz)
\end{align}
where we consider a $d$-dimensional Brownian motion and a $l$-dimensional compensated Poisson random process.
The coefficients $b: [0,T]\times \R^n \times A \to \R^n$, $\sigma: [0,T]\times \R^n \times A \to \R^{n\times d}$, 
$\gamma: [0,T]\times \R^n \times A \times \R^n \to \R^{n\times l}$ are continuous functions with respect to $(t,X_{t-},\alpha_t)$ and $\gamma(t,X_{t-},\alpha_t,\cdot)$ is also bounded
uniformly in $\alpha \in \mathcal{A}$ in any neighborhood of $z=0$. 

The set $\mathcal{A}$ is the set of all adapted and progressively measurable processes $\alpha: [0,T]\times \Omega \to  A$ with $A$ compact, such that:
$$ \E \biggl[ \int_0^T (\alpha_t)^m dt \biggr] < \infty \quad \mbox{ for } \quad m=1,2. $$

The process in (\ref{controlled_SDE}) is a general jump-diffusion process with coefficients depending on time, state and on the control $\alpha_t=a \in A$. 
In Chapter (\ref{Chapter1}), Section \ref{existence_uniqueness} we presented existence and uniqueness conditions for a time-homogeneous SDE. Thanks to \cite{Skorohod}, we can extend 
the conditions for the SDE in (\ref{controlled_SDE}).
Assume again the existence of a function $\rho : \R^n\to \R $ such that
\begin{equation}\label{rho}
 \int_{\R^n} |\rho(z)|^2 \nu(dz) < \infty.
\end{equation}
We have the following:
\begin{itemize}
 \item[(C1)] \textbf{Lipschitz condition} There exist $K >0$, such that $\forall x,y \in \R^n$, $\forall t \in [0,T]$ and $\forall a \in A$
 \begin{align}\label{Lipschitz_control}
  &|b(t,x,a) - b(t,y,a)| + || \sigma(t,x,a) - \sigma(t,y,a) || \leq K|x-y|,\\  
  & |\gamma(t,x,a,z) - \gamma(t,y,a,z)| \leq |\rho(z)|\,|x-y|. \label{Lipschitz_control2}
 \end{align}

 \item[(C2)] \textbf{Linear growth condition}. There exist $K>0$, such that $\forall x \in \R^n$,
 \begin{align}\label{Growth_control}
  &|b(t,x,a)|^2 + ||\sigma(t,x,a)||^2 \leq K (1+|x|^2).\\ 
  & |\gamma(t,x,a,z)| \leq |\rho(z)| (1+|x|). \label{Growth_control2}
 \end{align}
\end{itemize}
\begin{Definition}
 Let us define $\mathcal{T}_{t_0,T}$ as the set of all stopping times in $[t_0,T]$.
\end{Definition}
\begin{Theorem}
 Let (\ref{Lipschitz_control}),(\ref{Lipschitz_control2}),(\ref{Growth_control}) and (\ref{Growth_control2}) hold. 
 For any $k \in [0,2]$, there exist $C>0$ such that $\forall h,t \in [0,T]$, $x,y \in \R^n$, 
 $\alpha \in \mathcal{A}$ and $\tau \in \mathcal{T}_{0,h}$
 \begin{align}
  \E_{t,x}\bigl[ |X_{\tau}|^k \bigr] &\leq C (1+|x|^k) \\ \label{cond_2}
  \E_{t,x}\bigl[ |X_{\tau} - x|^k \bigr] &\leq C (1+|x|^k)h^{k/2} \\ \label{cond_3}
  \E_{t,x}\biggl[ \biggl( \sup_{0\leq s\leq h} |X_{s} - x| \biggr)^k \biggr] &\leq C (1+|x|^k)h^{k/2} \\ 
  \E_{t,x,y}\bigl[ |X_{\tau} - Y_{\tau}|^k \bigr] &\leq C |x-y|^2, \\ \nonumber
 \end{align}
\end{Theorem}
where we used the simple notation $\E_{t,x}[\cdot]$ to indicate the expectation conditioned on the initial value $X_t=x$.   
A detailed proof of the previous theorem can be found in the appendix of \cite{Ph98}.

Let us consider the space $\mathcal{C}_q$ defined in \ref{Cp}. We want to analyze the convergence properties of two integral operators:
We define the integral operators that appear in the infinitesimal generator of a controlled SDE of type (\ref{controlled_SDE}).
For $\phi \in C^{1,2}([0,T] \times \R^n) $ consider the operator:  
\begin{equation}
 \mathcal{I}^{1,\delta,a}(t,x,\phi) = \int_{|z|<\delta}
\biggl[ \phi(t,x + \gamma(t,x,a,z)) - \phi(t,x) - \gamma(t,x,a,z) \cdot D_x \phi(t,x) \biggr] \nu(dz)
\end{equation}
where $D_x \phi$ corresponds to the gradient of $\phi$.
For $\phi \in \mathcal{C}_2([0,T] \times \R^n) $ and $p \in \R^n$, consider the operator:
\begin{equation}
\mathcal{I}^{2,\delta,a}(t,x,p,\phi) = \int_{|z|\geq \delta}
[ \phi(t,x+ \gamma(t,x,a,z)) - \phi(t,x) - p \cdot \gamma(t,x,a,z)] \nu(dz)
\end{equation}
\begin{Theorem}
 The integral operators $\mathcal{I}^{1,\delta,a}$ and $\mathcal{I}^{2,\delta,a}$ converge and are uniformly bounded in $a$.
\end{Theorem}
\begin{proof}
 Consider the one dimensional case. For a general proof see \cite{Ph98}.
 For $|z|<\delta$, using a first order Taylor approximation on $\phi(t,x + \gamma(t,x,a,z))$, we obtain:
 \begin{align*}
 & \bigg|\phi(t,x + \gamma(t,x,a,z)) - \phi(t,x) - \gamma(t,x,a,z) \frac{\partial \phi(t,x)}{\partial x} \bigg| \\ 
 & \leq \gamma(t,x,a,z)^2 \sup_{y \in B(x,\delta)} \bigg| \frac{\partial^2 \phi(t,y)}{\partial y^2}\bigg| 
 \end{align*}
 Using the linear growth condition (\ref{Growth_control2}) and the equation (\ref{rho}) concludes the proof.\\
 For $|z|\geq \delta$, and $\phi \in \mathcal{C}_2([0,T] \times \R) $, all the terms in the integral are bounded. Thanks to the moment condition (assumption \textbf{M} 
 in Chapter \ref{Chapter1}) and to the equations (\ref{Growth_control2}) and (\ref{rho}), we can conclude the proof.
\end{proof}
Finally for $\phi \in C^{1,2}([0,T] \times \R^n) \bigcap \mathcal{C}_2([0,T] \times \R^n)$ we can define the 
\textbf{Integral operator} for $p=D_x \phi(t,x)$:
\begin{align}\label{int_oper}
 \mathcal{I}^{a}(t,x,\phi) &= \mathcal{I}^{1,\delta,a}(t,x,\phi) + \mathcal{I}^{2,\delta,a}(t,x,D_x \phi,\phi)\\ \nonumber
 &= \int_{\R}
\biggl[ \phi(t,x + \gamma(t,x,a,z)) - \phi(t,x) - \gamma(t,x,a,z) D_x \phi(t,x) \biggr] \nu(dz), 
\end{align}
and the integral operator in the generator (\ref{RN_log_gen}), is a special one-dimensional case with $\gamma(t,x,a,z) = z$.

\noindent
The general \textbf{Lévy operator} can be obtained with $\gamma(t,x,a,z) = z$, $\delta = 1$ and $p=0$ in this way:
\begin{align*}
\mathcal{I}_L(t,x,\phi) &= \mathcal{I}^{1,1,0}(t,x,\phi) + \mathcal{I}^{2,1,0}(t,x,0,\phi)\\  
                   &= \int_{\R} \biggl[ \phi(t,x+ z) - \phi(t,x) - z D_x \phi(t,x) 1_{|z|<1}(z) \biggr] \nu(dz) 
\end{align*}
The \textbf{integro-differential operator} is defined as:
\begin{equation}\label{int_diff_oper}
 \LL^{a}(t,x,\phi) = A^{a} \phi(t,x) + \mathcal{I}^{a}(t,x,\phi)
\end{equation}
with 
$$A^{a} \phi(t,x) = b(t,x,a) D_x \phi(t,x) + \frac{1}{2} \mbox{Tr} \biggl( \sigma(t,x,a)\sigma^T(t,x,a) \; D_x^2 \phi(t,x) \biggr) $$

\subsection{Dynamic Programming Principle}

Define the cylindrical region $Q = [t_0,T) \times \mathcal{O}$, with $\mathcal{O} \subseteq \R^n$ an open set. With abuse of notation we define the first exit time from $Q$ as 
\begin{equation}\label{exit_time_def}
 \tau = \inf \{ s: (s,X_s) \not\in Q \}. 
\end{equation}
Note that $\tau$ is the exit time from $O$ if $X_s$ exits before time $T$. If $X_s \in \OO$ $\forall s \in [t_0,T)$ than $\tau = T$. If $\OO = \R^n$, than the set $Q$ 
does not have lateral boundaries but only terminal boundary at $\{ T \} \times R^n$.

\noindent
Let $f:[0,T]\times \R^n \times A \to \R$ and $g: \bigl( [0,T)\times \R^n \backslash \mathcal{O} \bigr) \bigcup \bigl( \R^n \times \{T\} \bigr) \to \R$ two continuous functions. 
We assume that
\begin{enumerate}
 \item $g$ is lower bounded.
 \item For $C > 0$, they both satisfy the conditions:
 \begin{equation}\label{Lipschitz_f}
  |f(t,x,a) - f(s,y,a)| + |g(t,x)-g(s,y)| \leq C (|t-s|+|x-y|).
 \end{equation}
 \begin{equation}\label{linear_growth_f}
  |f(t,x,a)| + |g(t,x)| \leq C (1+|x|).
 \end{equation}
\end{enumerate}
Denote by $\mathcal{A}_{t,x}$ the non-empty subset of $\mathcal{A}$ dependent on the current state $(t,x)$ such that $\E_{t,x} [\int_{t}^{\tau} f(s,X_s,\alpha_s) ds] < \infty $. 
We define the \textbf{cost functional}
\begin{equation}\label{cost_functional}
 J(t,x;\alpha) = \E_{t,x} \biggl[ \int_t^{\tau} f(s,X_s,\alpha) ds + g(\tau, X_{\tau}) \biggr] . 
\end{equation}
Thanks to the growth condition \ref{linear_growth_f} and to equation (\ref{cond_3}), we can prove easily that the cost functional is well defined. 
The objective is to maximize the cost function over the set of control processes $\mathcal{A}_{t,x}$. We introduce the \textbf{value function}
\begin{equation}\label{general_value_function}
 V(t,x) = \sup_{\alpha \in \mathcal{A}_{t,x}} J(t,x,\alpha).
\end{equation}
If the optimal control $\hat \alpha$ exists, we have $V(t,x) = J(t,x,\hat \alpha)$.
In this thesis we only consider Markovian controls in the form $\alpha_s \in \mathcal{A}$ such that $\alpha_s = a(s,X_s)$ with $a \in A$. We also always assume that the value function is 
measurable.
\newline

\noindent
The dynamic programming principle (DPP) is a fundamental principle in the theory of stochastic control. It is formulated as follows:
\begin{Theorem} Dynamic Programming Principle:
 \begin{itemize}
  \item For all $\alpha \in \mathcal{A}_{t,x}$ and all stopping time $\theta \in \mathcal{T}_{t,\tau}$:
  \begin{equation}\label{DPP1}
   V(t,x) \geq \E_{t,x} \biggl[ \int_t^{\theta} f(s,X_s,\alpha) ds + V(\theta, X_{\theta}) \biggr]. 
  \end{equation}
  \item For all $\epsilon > 0$, there exists $\alpha \in \mathcal{A}_{t,x}$ such that $\forall \theta \in \mathcal{T}_{t,\tau}$: 
  \begin{equation}\label{DPP2}
      V(t,x) \leq \E_{t,x} \biggl[ \int_t^{\theta} f(s,X_s,\alpha) ds + V(\theta, X_{\theta}) \biggr] + \epsilon. 
  \end{equation}
 \end{itemize}
 Since $\epsilon$ is arbitrary, we can write the DPP in the following form:
  \begin{equation}\label{DPP3}
   V(t,x) = \sup_{\alpha \in \mathcal{A}_{t,x}} \E_{t,x} \biggl[ \int_t^{\theta} f(s,X_s,\alpha) ds + V(\theta, X_{\theta}) \biggr]. 
  \end{equation}
\end{Theorem}
The proof of the DPP is quite technical. Good references are \cite{Pham} and \cite{FlemingSoner}.


\noindent
An admissible control $\alpha \in \mathcal{A}_{t,x}$ is $\epsilon$-optimal for the initial condition $(t,x)$ if and only if it is $\epsilon$-optimal 
on every $\mathcal{A}_{\theta, X_{\theta}}$ with 
$\theta \in \mathcal{T}_{t,\tau}$. In order to determine the $\epsilon$-optimal control $\alpha_t$, it suffices to consider the DPP with a stopping time $\theta$
arbitrarily close to $t$.

\subsection{HJB equation}

In this section we assume that the value function is continuously differentiable and obtain a formal nonlinear PIDE satisfied by the value function.
In general however, the value function is not necessarily differentiable, and the notion of \emph{viscosity solution} should be considered.
The Hamilton-Jacobi-Bellman equation (HJB) is the infinitesimal version of the DPP. It describes the local behavior of the value function when the stopping time $\theta$ converges to $t$. 
The HJB equation is also called dynamic programming equation.
\newline

\noindent
Let us consider the stopping time $\theta = t + h$ in the DPP (\ref{DPP3}) and a constant control $\alpha_s = a$. 
Subtract $V(t,x)$ from both sides and then divide by $h$. This yields
\begin{align*}
   0 =& \sup_{a \in A} \E_{t,x} \biggl[ \int_t^{t+h} f(s,X_s,a) ds + V(t+h, X_{t+h}) - V(t,x) \biggr]. \\ 
     =& \sup_{a \in A} \E_{t,x} \biggl[ \frac{1}{h} \int_t^{t+h} f(s,X_s,a) 
     + \biggl( \frac{\partial V}{\partial s} + \LL^{a}V \biggr)(s, X_{s}) \, ds \biggr]. \\
\end{align*}
where we used the Dynkin formula (Eq. \ref{Dynkin_formula}) assuming $V \in C^{1,2}$, with $\LL^{\alpha}$ the integro-differential operator \ref{int_diff_oper} 
for the constant control $a$.
By sending $h \to 0$ and using the mean value theorem we obtain the \textbf{dynamic programming equation}:
\begin{equation}\label{DPE}
 \frac{\partial V(t,x)}{\partial t} + \sup_{a \in A} \biggl( f(t,x,a) + \LL^{a}V(t,x)  \biggr) = 0.
\end{equation}
It is convenient to multiply by $-1$ the equation (for a clear interpretation of the viscosity solution) and to define the Hamiltonian function $\mathcal{H}$ 
that contains all the terms depending on the control $a$.
\begin{align} \nonumber
 0 = & \frac{\partial V(t,x)}{\partial t} + \sup_{a \in A} \biggl( f(t,x,a) + \LL^{a}V(t,x)  \biggr) \\ \nonumber
 0 = & - \frac{\partial V(t,x)}{\partial t} - \sup_{a \in A} \biggl( f(t,x,a) + \LL^{a}V(t,x)  \biggr) \\ \label{DPE2}
 0 = & - \frac{\partial V(t,x)}{\partial t} - \mathcal{H}\bigl( t,x,D_x V, D_{xx} V, \mathcal{I}^a(t,x,V) \bigr) 
\end{align}
where for $(t,x,p,M,I^a) \in Q \times \R^n \times S^n \times \R$:\footnote{$S^n$ is the set of symmetric matrices of dimension $n$.} 
\begin{align}\label{Hamiltonian}
 \mathcal{H}(t,x,p,M,I^a) =& \sup_{a \in A} \biggl( f(t,x,a) \, + b(t,x,a) p \\ \nonumber
              &+ \frac{1}{2} \mbox{Tr} \bigl( \sigma(t,x,a)\sigma^T(t,x,a) \, M \bigr) +I^a \biggr) 
\end{align}
The regular terminal and lateral boundary conditions of this PIDE are: 
\begin{equation}\label{DPE_term_cond}
 V(T,x) = g(T,x).
\end{equation}
\begin{equation}\label{DPE_lateral_cond}
 V(t,x) = g(t,x) \quad \mbox{ for } \quad (t,x) \in [0,T)\times (\R^n \backslash \mathcal{O}). 
\end{equation}


\subsection{Singular control}

When the control space $A$ is unbounded, the Hamiltonian (\ref{Hamiltonian}) may take infinite values. Let us assume that $A$ is a closed cone in $\R^n$ such that 
\begin{equation}\label{unbounded_set}
 a \in A, \quad \lambda > 0 \Longrightarrow \lambda a \in A.
\end{equation}
Let us assume also that the control influences linearly the dynamics of the system and the running cost function.
\begin{align}\label{linear_control}
 & b(t,x,\alpha_t) = \hat b(t,x) + \alpha_t, \quad \sigma(t,x,\alpha_t) = \hat \sigma(t,x), \\ 
 & \hat \gamma(t,x,\alpha_t,z) = \gamma(t,x,z), \quad f(t,x,\alpha_t) = \hat f(t,x) + \alpha_t. 
\end{align}
The Hamiltonian becomes
\begin{align*}
 \mathcal{H}(t,x,p,M,I^a) =& \biggl( \hat f(t,x) \, + \hat b(t,x) p \\ \nonumber
              &+ \frac{1}{2} \mbox{Tr} \bigl( \sigma(t,x)\sigma^T(t,x) \, M \bigr) +\hat I \biggr) + \underbrace{\sup_{a \in A} \biggl( a(p + 1) \biggr) }_{\hat H(p)} 
\end{align*}
If for some $a\in A$ we have $a(1+p)>0$, then $\hat H(p) = \infty$. We can define 
\begin{equation}
 H(p) = \sup_{a \in \hat K} \biggl( a(p + 1) \biggr) \quad \mbox{ with } \quad \hat K = \{ a \in A: |a|=1 \}
\end{equation}
such that
\begin{equation}
\hat H(p) = \begin{cases} 
 \infty, & \mbox{if } H(p) >0 \\ 
  0,     & \mbox{if } H(p) \leq 0 . 
\end{cases} 
\end{equation}
Using this last equation together with the HJB (\ref{DPE}) with $a=0$ we have the following two equations:
\begin{equation}
\begin{cases}
 \frac{\partial V(t,x)}{\partial t} + \hat f(t,x) + \LL^{a=0}V(t,x)  \leq 0. \\
 H\bigl( D_x V(t,x) \bigr) \leq 0.
\end{cases}
\end{equation}
Now, suppose $H(D_x V(t,x)) < 0$. Then $\hat H(p)=0$ and the optimal control is indeed $a=0$, with the uncontrolled HJB equal to zero:
\begin{equation}
 H\bigl( D_x V(t,x) \bigr) < 0 \quad \Longrightarrow \quad  \frac{\partial V(t,x)}{\partial t} + \hat f(t,x) + \LL^{a=0}V(t,x) = 0.
\end{equation}
The last equation can be written in a more compact form. We have the following \textbf{variational inequality}:
\begin{equation}\label{variational_inequality}
 \max \biggl\{ \frac{\partial V(t,x)}{\partial t} + \hat f(t,x) + \LL^{a=0}V(t,x) \,,\, H\bigl( D_x V(t,x) \bigr) \biggr \} =0 \quad \mbox{for} \quad (t,x)\in Q. 
\end{equation}







\section{Viscosity solutions theory}


This section is dedicated to the definition of viscosity solutions.
Here I follow closely the theory presented in \cite{Cont}, \cite{Pham} and \cite{BaIm08}.
In the general discontinuous viscosity solutions approach, there is no
need to prove a priori the continuity of the value function $V$. The continuity will actually follow from
a strong comparison principle. Here I present the general theory for semi-continuous value functions following \cite{Cont}.
Recall the definition of semi-continuous function:
\begin{Definition}
 Suppose $\mathcal{X}$ is a topological space and $x_0$ is a point in $\mathcal{X}$.
 The function $f: \mathcal{X} \to \R$ is \textbf{upper semi-continuous} (USC) in $x_0$ if for every $\epsilon > 0$ exists a neighborhood of $x_0$, $U_{x_0}$, such that
 $$ f(x) \leq f(x_0) + \epsilon \quad \quad \forall x \in U_{x_0}. $$
\end{Definition}
\begin{Definition}
 Under the same assumptions,
 the function $f: \mathcal{X} \to \R$ is \textbf{lower semi-continuous} (LSC) in $x_0$ if for every $\epsilon > 0$ exists a neighborhood of $x_0$, $U_{x_0}$, such that
 $$ f(x) \geq f(x_0) - \epsilon \quad \quad \forall x \in U_{x_0}. $$
\end{Definition}



\subsection{Definition viscosity solution}


Let us consider a general parabolic problem:
\begin{equation}\label{parabolic_PIDE}
\begin{cases}
F(t,x,u,D_t u,D_x u,D_{xx}u,\I(t,x,u)) = 0 \quad \mbox{ for } \quad (t,x) \in Q \\
u(t,x) = g(t,x) \quad \mbox{ for } \quad (t,x) \not \in Q
\end{cases}
\end{equation}
where $g \in C^0 \bigcap \mathcal{C}_2([0,T] \times (\R\backslash \OO))$ is a given function and 
$F$ is a continuous function that satisfies the following elliptic/parabolic local and non local conditions. 
For all $t\in [0,T); \; x \in \OO; \; r,\hat r \in \R; \; q,\hat q \in \R; \; 
p \in \R^n; \; M,\hat M \in \SI^n; \; \I, \hat \I \in \R$:
\begin{itemize}
 \item $r \leq \hat r \; \Longrightarrow  \; F(t,x,r,q,p,M,\I) \leq F(t,x,\hat r,q,p,M,\I) $
 \item $q \leq \hat q \; \Longrightarrow  \; F(t,x,r,q,p,M,\I) \geq F(t,x,r,\hat q,p,M,\I) $
 \item $M \leq \hat M \; \Longrightarrow  \; F(t,x,r,q,p,M,\I) \geq F(t,x,r,q,p,\hat M,\I) $
 \item $\I \leq \hat \I \; \Longrightarrow  \; F(t,x,r,q,p,M,\I) \geq F(t,x,r,q,p,M,\hat \I). $
\end{itemize}
where the matrix ordering is intended with this meaning:
$$ \hat M \geq M \Leftrightarrow \hat M - M \mbox{ is positive semi-definite.} $$
We can now introduce the viscosity solution definition:
\begin{Definition}
 An USC function $u$ is a \textbf{viscosity subsolution} of (\ref{parabolic_PIDE})
 if for any $(\bar t, \bar x) \in [0,T]\times \R^n$ and any test function $ \phi \in C^{1,2}([0,T] \times \R^n) \bigcap \mathcal{C}_2([0,T] \times \R^n)$ 
 such that $u-\phi$ has a global maximum at $(\bar t,\bar x)$ the following is satisfied:
\begin{align}\label{subsolution}
 & F\biggl( \bar t,\bar x,u(\bar t,\bar x),D_t \phi(\bar t,\bar x), D_x \phi(\bar t,\bar x),D_{xx}\phi(\bar t,\bar x),\I(\bar t, \bar x,\phi(\bar t,\bar x))\biggr) \leq 0 \\ \nonumber
 &  \quad \quad \mbox{ for } \quad (\bar t,\bar x) \in Q, \\ \nonumber
 & \min \biggl\{ F \biggl( \bar t,\bar x,u(\bar t,\bar x), D_t \phi(\bar t,\bar x), D_x \phi(\bar t,\bar x), 
 D_{xx}\phi(\bar t,\bar x),\I(\bar t, \bar x,\phi(\bar t,\bar x)) \biggr) ,\\ \nonumber 
 & \quad \quad u(\bar t,\bar x) - g(\bar t,\bar x) \biggr\} \leq 0  \quad  \mbox{ for } \quad (\bar t,\bar x) \in \partial Q, \\ \nonumber
 & u(\bar t,\bar x) - g(\bar t,\bar x) \leq 0  \quad  \mbox{ for } \quad (\bar t,\bar x) \not \in \bar Q.
\end{align}
\end{Definition}
In the same way we define:
\begin{Definition}
 A LSC function $u$ is a \textbf{viscosity supersolution} of (\ref{parabolic_PIDE})
 if for any $(\bar t, \bar x) \in [0,T]\times \R^n$ and any test functions $ \phi \in C^{1,2}([0,T] \times \R^n) \bigcap \mathcal{C}_2([0,T] \times \R^n)$ 
 such that $u-\phi$ has a global minimum at $(\bar t,\bar x)$ the following is satisfied:
\begin{align}\label{supersolution}
 & F\biggl( \bar t,\bar x,u(\bar t,\bar x),D_t \phi(\bar t,\bar x), D_x \phi(\bar t,\bar x),D_{xx}\phi(\bar t,\bar x),\I(\bar t, \bar x,\phi(\bar t,\bar x))\biggr) \geq 0 \\ \nonumber
 &  \quad \quad \mbox{ for } \quad (\bar t,\bar x) \in Q, \\ \nonumber
 & \max \biggl\{ F \biggl( \bar t,\bar x,u(\bar t,\bar x), D_t \phi(\bar t,\bar x), D_x \phi(\bar t,\bar x), 
 D_{xx}\phi(\bar t,\bar x),\I(\bar t, \bar x,\phi(\bar t,\bar x)) \biggr) ,\\ \nonumber 
 & \quad \quad u(\bar t,\bar x) - g(\bar t,\bar x) \biggr\} \geq 0  \quad  \mbox{ for } \quad (\bar t,\bar x) \in \partial Q, \\ \nonumber
 & u(\bar t,\bar x) - g(\bar t,\bar x) \geq 0  \quad  \mbox{ for } \quad (\bar t,\bar x) \not \in \bar Q.
\end{align}
\end{Definition}

When a function is both a viscosity subsolution and supersolution, it is called a \textbf{viscosity solution}.


