

\chapter{HJB equation and viscosity solution}\label{Chapter4}
%\blindtext
\minitoc% Creating an actual minitoc

\vspace{5em}

In this chapter we present a brief introduction to the use of the \textbf{dynamic programming principle} for solving stochastic control problems. 
The fundamental equation of dynamic programming is a nonlinear evolution equation for the \textbf{value function}.
The value function is the optimum value of the payoff considered as a function of the initial data.
This principle was introduced in the 1950s by Bellman, see \cite{Bellman}. 
For controlled Lévy processes,
the approach yields a certain nonlinear PIDE, usually of first or second order, called Hamilton-Jacobi-Bellman (HJB) equation. 
Under appropriate conditions the value function satisfies the HJB equation.

The theory of viscosity solutions, provides a convenient framework in which to study HJB equations.
Typically, the value function is not smooth enough to satisfy the HJB in the classical sense. However, under certain assumptions, it is the unique viscosity
solution of the HJB equation with appropriate boundary conditions. For a general review of the theory of viscosity solution see \cite{CIL92}.

The first notion of viscosity solution has been introduced by \cite{CL83} for first order partial differential equations. The theory has
been immediately extended by \cite{PLL83} for second order PDEs.
Uniqueness results for general second order equations are presented in \cite{Je88}, \cite{Is89} and \cite{IsLi90}. 
The main development of these works is the \emph{Ishii's lemma}, which plays a key role in most of the uniqueness proofs.
The viscosity solution theory has been further extended by \cite{Soner86b}, \cite{Soner86} and \cite{Sayah91} for piecewise deterministic processes with random jumps.
An important paper for viscosity solution of Lévy-type PIDEs (PIDEs involving the generator of a Lévy process) is \cite{BaIm08}, where the Ishii's
lemma is extended. 
Other important works that deserve to be mentioned are \cite{Ph98}, that analyzes HJB equations for optimal stopping problems under Lévy processes and \cite{CoVo05} that analyze
the solution of linear PIDEs derived from option prices problems (plain vanilla and barrier options) for finite and infinite activity Lévy processes.
\vspace{1em}

\section{Optimal control framework}\label{Optimal_control_framework}

We consider a framework where the state of the system is governed by the controlled SDE with values in $\R^n$:
\begin{align}\label{controlled_SDE}
 dX_t =& b(t,X_{t-},\alpha_t) dt + \sigma (t,X_{t-},\alpha_t) dW_t \\ \nonumber
      &+ \int_{\R} \gamma(t,X_{t-},\alpha_t,z) \tilde N(dt,dz).
\end{align}
where we consider a $d$-dimensional Brownian motion and a $l$-dimensional compensated Poisson random measure. 
The coefficients $b: [0,T]\times \R^n \times A \to \R^n$, $\sigma: [0,T]\times \R^n \times A \to \R^{n \times d}$, 
$\gamma: [0,T]\times \R^n \times A \times \R^n \to \R^{n \times l}$ are continuous functions with respect to $(t,x,a)$ and $\gamma(t,x,a,\cdot)$ is also bounded
uniformly in $a \in A$ in any neighborhood of $z=0$. 
In order to simplify the notation, all the indexes in the vectorial equation (\ref{controlled_SDE}) are suppressed 
(see Theorem 1.16 in \cite{OksendalSulem} for the description of the indexed equation).

The set $\mathcal{A}$ is the set of all progressively measurable processes $\alpha: [0,T]\times \Omega \to  A$ with $A$ compact.
In this thesis we only consider Markovian controls such that $\alpha_t := \alpha(t,X_{t^-}) \to a \in A$, with $t\in [0,T]$.

The process in (\ref{controlled_SDE}) is a generalized jump-diffusion process with coefficients depending on time, state and on the control. 
In Chapter (\ref{Chapter1}), Section \ref{existence_uniqueness} we presented existence and uniqueness conditions for a time-homogeneous SDE. Following \cite{Skorohod}, we can extend 
the conditions for the SDE in (\ref{controlled_SDE}).
Let us consider a function $\rho : \R^n \to \R $ such that
\begin{equation}\label{rho}
 \int_{\R} |\rho(z)|^2 \nu(dz) < \infty.
\end{equation}
We have the following:
\begin{itemize}
 \item[(C1)] \textbf{Lipschitz condition} There exist $K >0$, such that $\forall x,y \in \R^n$, $\forall t \in [0,T]$ and $\forall a \in A$
 \begin{align}\label{Lipschitz_control}
  &|b(t,x,a) - b(t,y,a)| + || \sigma(t,x,a) - \sigma(t,y,a) || \leq K |x-y|,\\  
  & ||\gamma(t,x,a,z) - \gamma(t,y,a,z)|| \leq |\rho(z)|\,|x-y|. \label{Lipschitz_control2}
 \end{align}
\end{itemize}
\begin{Theorem}
 The assumptions (\ref{Lipschitz_control}) ensure that for each control $\alpha \in \mathcal{A}$, there exists a unique strong solution of (\ref{controlled_SDE}) with given initial 
 conditions.
\end{Theorem}
See Chapter 3 of \cite{Skorohod}.

Note that the Lipschitz conditions (\ref{Lipschitz_control}) and the continuity of $b$, $\sigma$, $\gamma$, imply the growth conditions:
\begin{align}
 & |b(t,x,a)| + ||\sigma(t,x,a)||  \leq K |1+|x|| \\  
 & || \gamma(t,x,a,z) || \leq \rho(z) (1+|x|) \label{Growth_control2}
\end{align}


\begin{Definition}
 Let us define $\mathcal{T}_{t_1,t_2}$ as the set of all $\mathcal{F}_s$-stopping times for each $ s \in [t_1,t_2]$.
\end{Definition}
\begin{Theorem}
 Let us consider the process described by (\ref{controlled_SDE}) and the hypothesis (\ref{Lipschitz_control}) and (\ref{Lipschitz_control2}) satisfied. 
 For any $k \in [0,2]$, there exist $C>0$ such that $\forall h,t \in [0,T]$, $x,y \in \R^n$, 
 $\alpha \in \mathcal{A}$ and $\tau \in \mathcal{T}_{0,h}$
 \begin{align}
  \E_{t,x}\bigl[ |X_{\tau}|^k \bigr] &\leq C (1+|x|^k) \\ \label{cond_2}
  \E_{t,x}\bigl[ |X_{\tau} - x|^k \bigr] &\leq C (1+|x|^k)h^{k/2} \\ \label{cond_3}
  \E_{t,x}\biggl[ \biggl( \sup_{0\leq s\leq h} |X_{s} - x| \biggr)^k \biggr] &\leq C (1+|x|^k)h^{k/2} \\ 
  \E_{t,x,y}\bigl[ |X_{\tau} - Y_{\tau}|^k \bigr] &\leq C |x-y|^2 \\ \nonumber
 \end{align}
\end{Theorem}
where we used the simple notation $\E_{t,x}[\cdot]$ to indicate the expectation conditioned on the initial value $X_t=x$.   
A detailed proof of the previous theorem can be found in the appendix of \cite{Ph98}. \\

\noindent
For $\delta > 0$, let us define the following two useful integral operators.

For $\phi \in C^{1,2}([0,T] \times \R^n) $ consider the operator:\footnote{Inside the operators $\mathcal{I}^{1,\delta,a}$ and $\mathcal{I}^{1,\delta,a}$, 
the scalar product considers only the $k$-column (for $1\leq k \leq l$) of the $n\times l$ matrix $\gamma(t,x,a,z)$. As explained in the beginning of the section, 
we suppressed the indexes to simplify the notation. }
\begin{equation}
 \mathcal{I}^{1,\delta,a}(t,x,\phi) = \int_{|z|<\delta}
\biggl[ \phi(t,x + \gamma(t,x,a,z)) - \phi(t,x) - D_x \phi(t,x) \cdot \gamma(t,x,a,z) \biggr] \nu(dz)
\end{equation}
where $D_x \phi$ corresponds to the gradient of $\phi$.\\
For $\phi \in \mathcal{C}_2([0,T] \times \R^n) $ (the space $\mathcal{C}_q$ was defined in \ref{Cp}) and $p \in \R^n$, consider the operator:
\begin{equation}
\mathcal{I}^{2,\delta,a}(t,x,p,\phi) = \int_{|z|\geq \delta}
[ \phi(t,x+ \gamma(t,x,a,z)) - \phi(t,x) - p \cdot \gamma(t,x,a,z)] \nu(dz)
\end{equation}
We can check that the integral operators $\mathcal{I}^{1,\delta,a}$ and $\mathcal{I}^{2,\delta,a}$ are well defined.

For $|z|<\delta$, using a first order Taylor approximation on $\phi(t,x + \gamma(t,x,a,z))$, we obtain:
\begin{align*}
 & \bigg|\phi(t,x + \gamma(t,x,a,z)) - \phi(t,x) - D_x \phi(t,x) \cdot \gamma(t,x,a,z) \bigg| \\ 
 & \leq \bigg| \frac{1}{2} \gamma(t,x,a,z)^T \cdot \sup_{y \in B(x,\rho(z)(1+|x|) )}  D_{xx} \phi(t,y) \cdot \gamma(t,x,a,z) \bigg| 
\end{align*}
Using the growth condition (\ref{Growth_control2}) and the equation (\ref{rho}) we can see that the integral is well defined.

For $|z|\geq \delta$, by definition $\phi \in \mathcal{C}_2([0,T] \times \R^n) $ have quadratic growth and therefore are bounded by 
the moment condition (assumption \textbf{M} 
in Chapter \ref{Chapter1}). Using equations (\ref{Growth_control2}) and (\ref{rho}), the verification is completed.\\

\noindent
Finally, for $\phi \in C^{1,2}([0,T] \times \R^n) \bigcap \mathcal{C}_2([0,T] \times \R^n)$ we can define the 
\textbf{Integral operator}:
\begin{align}\label{int_oper}
 \mathcal{I}^{a}(t,x,\phi) &= \mathcal{I}^{1,\delta,a}(t,x,\phi) + \mathcal{I}^{2,\delta,a}(t,x,D_x \phi,\phi)\\ \nonumber
 &= \int_{\R^n}
\biggl[ \phi(t,x + \gamma(t,x,a,z)) - \phi(t,x) - D_x \phi(t,x) \cdot \gamma(t,x,a,z) \biggr] \nu(dz), 
\end{align}
and the integral operator in the generator (\ref{RN_log_gen}), is a special case with $\gamma(t,x,a,z) = z$.

\noindent
The 1 dimensional \textbf{Lévy operator} can be obtained with $\gamma(t,x,a,z) = z$, $\delta = 1$ and $p=0$ in this way:
\begin{align*}
\mathcal{I}_L(t,x,\phi) &= \mathcal{I}^{1,1,0}(t,x,\phi) + \mathcal{I}^{2,1,0}(t,x,0,\phi)\\  
                   &= \int_{\R} \biggl[ \phi(t,x+ z) - \phi(t,x) - z D_x \phi(t,x) \mathbbm{1}_{|z|<1}(z) \biggr] \nu(dz) 
\end{align*}
The \textbf{controlled integro-differential operator} is defined as:
\begin{equation}\label{int_diff_oper}
 \LL^{a}(t,x,\phi) = A^{a} \phi(t,x) + \mathcal{I}^{a}(t,x,\phi)
\end{equation}
with 
$$A^{a} \phi(t,x) = b(t,x,a) D_x \phi(t,x) + \frac{1}{2} \sigma(t,x,a) \; D_x^2 \phi(t,x) $$



\subsection{Dynamic Programming Principle}

Define the cylindrical region $Q = [t_0,T) \times \mathcal{O}$, with $\mathcal{O} \subseteq \R^n$ an open set. 
With abuse of notation we define the first exit time from $Q$ as 
\begin{equation}\label{exit_time_def}
 \tau = \inf \{ s: (s,X_s) \not\in Q \}. 
\end{equation}
Note that $\tau$ is the exit time from $O$ if $X_s$ exits before time $T$. If $X_s \in \OO$ $\forall s \in [t_0,T)$ than $\tau = T$. If $\OO = \R^n$, than the set $Q$ 
does not have lateral boundaries but only terminal boundary at $\{ T \} \times R^n$.

\noindent
Let $f:[0,T]\times \R^n \times A \to \R$ and $g: \bigl( [0,T)\times \R^n \backslash \mathcal{O} \bigr) \bigcup \bigl( \R^n \times \{T\} \bigr) \to \R$ two continuous functions. 
We assume that there exists $C < \infty$, they satisfy the condition:
% \begin{equation}\label{Lipschitz_f}
%  |f(t,x,a) - f(s,y,a)| + |g(t,x)-g(s,y)| \leq C (|t-s|+|x-y|).
% \end{equation}
 \begin{equation}\label{linear_growth_f}
  |f(t,x,a)| + |g(t,x)| \leq C (1+|x|^2) \quad \quad \forall a \in A.
 \end{equation}
Denote by $\mathcal{A}_{t,x}$ the subset of $\mathcal{A}$ dependent on the current state $(t,x)$ such that $\E_{t,x} [\int_{t}^{\tau} |f(s,X_s,\alpha_s)| ds] < \infty $. 
We define the \textbf{objective function}
\begin{equation}\label{cost_functional}
 J(t,x;\alpha) = \E_{t,x} \biggl[ \int_t^{\tau} f(s,X_s,\alpha_s) ds + g(\tau, X_{\tau}) \biggr] . 
\end{equation}
Thanks to the growth condition \ref{linear_growth_f} and to equation (\ref{cond_3}), we can verify easily that the objective function is well defined. 
We want to maximize the objective function over the set of control processes $\mathcal{A}_{t,x}$. We introduce the \textbf{value function}
\begin{equation}\label{general_value_function}
 V(t,x) = \sup_{\alpha \in \mathcal{A}_{t,x}} J(t,x,\alpha).
\end{equation}
If the optimal control $\hat \alpha$ exists, we have $V(t,x) = J(t,x,\hat \alpha)$.
We also always assume that the value function is 
measurable.
\newline

\noindent
The dynamic programming principle (DPP) is a fundamental principle in the theory of stochastic control. It is formulated as follows:
\begin{Theorem} \textbf{Dynamic Programming Principle}:
 \begin{itemize}
  \item For all $\alpha \in \mathcal{A}_{t,x}$ and all stopping time $\theta \in \mathcal{T}_{t,\tau}$:
  \begin{equation}\label{DPP1}
   V(t,x) \geq \E_{t,x} \biggl[ \int_t^{\theta} f(s,X_s,\alpha_s) ds + V(\theta, X_{\theta}) \biggr]. 
  \end{equation}
  \item For all $\epsilon > 0$, there exists $\alpha \in \mathcal{A}_{t,x}$ such that $\forall \theta \in \mathcal{T}_{t,\tau}$: 
  \begin{equation}\label{DPP2}
      V(t,x) \leq \E_{t,x} \biggl[ \int_t^{\theta} f(s,X_s,\alpha_s) ds + V(\theta, X_{\theta}) \biggr] + \epsilon. 
  \end{equation}
 \end{itemize}
 Since $\epsilon$ is arbitrary, we can write the DPP in the following form:
  \begin{equation}\label{DPP3}
   V(t,x) = \sup_{\alpha \in \mathcal{A}_{t,x}} \E_{t,x} \biggl[ \int_t^{\theta} f(s,X_s,\alpha_s) ds + V(\theta, X_{\theta}) \biggr]. 
  \end{equation}
\end{Theorem}
The proof of the DPP is quite technical and is presented in the main textbooks on stochastic control theory.   
A proof based on the assumptions presented above can be found in \cite{Gol16}.\\
However, since the main problem in this thesis (see Chapter \ref{Chapter5}) is a singular control problem (see Section \ref{singular_control}), 
some assumptions will be different from the previous general presentation.
We will need to refer to the specific proof of the DPP presented in \cite{Kab16}.    \\


\noindent
An admissible control $\alpha \in \mathcal{A}_{t,x}$ is $\epsilon$-optimal for the initial condition $(t,x)$ if and only if it is $\epsilon$-optimal 
on every $\mathcal{A}_{\theta, X_{\theta}}$ with 
$\theta \in \mathcal{T}_{t,\tau}$. In order to determine the $\epsilon$-optimal control $\alpha_t$, it suffices to consider the DPP with a stopping time $\theta$
arbitrarily close to $t$.

\subsection{HJB equation (formal derivation)}

In this section we assume that the value function is continuously differentiable and obtain a formal nonlinear PIDE satisfied by the value function.
In general however, the value function is not necessarily differentiable, and the notion of \emph{viscosity solution} should be considered.
The Hamilton-Jacobi-Bellman equation (HJB) is the infinitesimal version of the DPP. It describes the local behavior of the value function when the stopping time $\theta$ converges to $t$. 
The HJB equation is also called \textbf{dynamic programming equation}.
\newline

\noindent
Let us consider the stopping time $\theta = t + h$, with $h>0$, in the DPP (\ref{DPP1}) and a constant control $\alpha_s = a$ for $s \in [t,\theta]$. 
Subtract $V(t,x)$ from both sides and then divide by $h$. This yields
\begin{align*}
   0 \geq& \E_{t,x} \biggl[ \int_t^{t+h} f(s,X_s,a) ds + V(t+h, X_{t+h}) - V(t,x) \biggr]. \\ 
     \geq& \E_{t,x} \biggl[ \frac{1}{h} \int_t^{t+h} f(s,X_s,a) 
     + \biggl( \frac{\partial V}{\partial s} + \LL^{a}V \biggr)(s, X_{s}) \, ds \biggr]. \\
\end{align*}
where we used the Dynkin formula (Eq. \ref{Dynkin_formula}) assuming $V \in C^{1,2}$, with $\LL^{\alpha}$ the integro-differential operator \ref{int_diff_oper}.
By sending $h \to 0$ and using the mean value theorem we obtain the equation:
\begin{equation}
 \frac{\partial V(t,x)}{\partial t} + \biggl( f(t,x,a) + \LL^{a}V(t,x)  \biggr) \leq 0.
\end{equation}
Since this hold true for every $a \in A$, then:
\begin{equation}
 -\frac{\partial V(t,x)}{\partial t} - \sup_{a \in A} \biggl( f(t,x,a) + \LL^{a}V(t,x)  \biggr) \geq 0.
\end{equation}
On the other hand, suppose there exists the optimal control $\alpha^*$, then:
\begin{equation}
 0 = \E_{t,x} \biggl[ \int_t^{t+h} f(s,X^*_s,\alpha^*_s) ds + V(t+h, X^*_{t+h}) - V(t,x) \biggr],
\end{equation}
where $X^*$ is the optimal state, solution of (\ref{controlled_SDE}), with the control $\alpha^*$.
Using similar arguments as above we obtain:
\begin{equation}
 -\frac{\partial V(t,x)}{\partial t} - \biggl( f(t,x,\alpha_t^*) + \LL^{\alpha_t^*} V(t,x)  \biggr) = 0.
\end{equation}
Since $A$ is compact, and $b$, $\sigma$, $\gamma$ and $f$ are continuous, under these hypothesis the supremum is always attained.
We can combine the two results and write the \textbf{dynamic programming equation}:
\begin{equation}\label{DPE}
 -\frac{\partial V(t,x)}{\partial t} - \sup_{a \in A} \biggl( f(t,x,a) + \LL^{a}V(t,x)  \biggr) = 0.
\end{equation}
Let us define the Hamiltonian function
\begin{align} \nonumber
 0 = & - \frac{\partial V(t,x)}{\partial t} - \mathcal{H}\bigl( t,x,D_x V, D_{xx} V, \mathcal{I}^a(t,x,V) \bigr) 
\end{align}
where for $(t,x,p,M,I^a) \in Q \times \R^n \times S^n \times \R$:\footnote{$S^n$ is the set of symmetric matrices of dimension $n$.} 
\begin{align}\label{Hamiltonian}
 \mathcal{H}(t,x,p,M,I^a) =& \sup_{a \in A} \biggl( f(t,x,a) \, + b(t,x,a) p \\ \nonumber
              &+ \frac{1}{2} \mbox{Tr} \bigl( \sigma(t,x,a)\sigma^T(t,x,a) \, M \bigr) +I^a \biggr) 
\end{align}
The regular terminal and lateral boundary conditions of this PIDE are: 
\begin{equation}\label{DPE_term_cond}
 V(T,x) = g(T,x).
\end{equation}
\begin{equation}\label{DPE_lateral_cond}
 V(t,x) = g(t,x) \quad \mbox{ for } \quad (t,x) \in [0,T)\times (\R^n \backslash \mathcal{O}). 
\end{equation}


\section{Singular control}\label{singular_control}

In the previous section, the theory of stochastic control for generalized jump-diffusion processes is presented under the assumption that the control process 
takes values in a compact space.
In this section this assumption is relaxed, and the control space $A$ is assumed to be unbounded.
When the problem coefficients are linear functions of the control, the form (\ref{DPE}) for the HJB equation is no more valid. 
We will present a formal derivation of the HJB equation, which under these assumptions has the form of 
a \textbf{variational inequality}. 
A rigorous basis to this derivation can be given a posteriori by means of a verification theorem, or by viscosity solution theory 
for more details we refer to Chapter 8.4 of \cite{FlemingSoner} or to Theorem 5.2 of \cite{OksendalSulem}. \\ 
We then present the general formulation for singular control problems.
The name \emph{singular} comes from the fact that the control process can be singular with respect to the Lebesgue measure $dt$.

\subsection{Derivation of the variational inequality}

When the control space $A$ is unbounded, the Hamiltonian (\ref{Hamiltonian}) may take infinite values. Let us assume that $A$ is a closed cone in $\R^n$ i.e. 
\begin{equation}\label{unbounded_set}
 a \in A, \quad \lambda > 0 \Longrightarrow \lambda a \in A.
\end{equation}
Let us assume also that the control influences linearly the dynamics of the system and the objective function.
\begin{align}\label{linear_control}
 & b(t,x,a) = \hat b(t,x) + \kappa(t,x) a, \quad \sigma(t,x,a) = \hat \sigma(t,x), \\ 
 & \gamma(t,x,a,z) = \hat \gamma(t,x,z), \quad f(t,x,a) = \hat f(t,x) + h(t,x) a. 
\end{align}
The Hamiltonian becomes
\begin{align*}
 \mathcal{H}(t,x,p,M,I^a) =& \biggl( \hat f(t,x) \, + p \, \hat b(t,x) \\ \nonumber
              &+ \frac{1}{2} \mbox{Tr} \bigl( \sigma(t,x)\sigma^T(t,x) \, M \bigr) +\hat I \biggr) + 
              \underbrace{\sup_{a \in A} \biggl( \bigl( p \, \kappa(t,x) + h(t,x) \bigr) a \biggr) }_{\hat H(p)} 
\end{align*}
If for some $a\in A$ and for fixed $(t,x)$ we have $\bigl( p \, \kappa(t,x) + h(t,x) \bigr) a>0$, then $\hat H(p) = \infty$. We can define 
\begin{equation}
 H(p) = \sup_{a \in \hat K} \biggl( \bigl( p \, \kappa(t,x) + h(t,x) \bigr) a \biggr) \quad \mbox{ with } \quad \hat K = \{ a \in A: |a|=1 \}
\end{equation}
such that
\begin{equation}
\hat H(p) = \begin{cases} 
 \infty, & \mbox{if } H(p) >0 \\ 
  0,     & \mbox{if } H(p) \leq 0 . 
\end{cases} 
\end{equation}
Using this last equation together with the HJB (\ref{DPE}) with $a=0$ we have the following two equations:
\begin{equation}
\begin{cases}
 \frac{\partial V(t,x)}{\partial t} + \hat f(t,x) + \LL^{a=0}V(t,x)  \leq 0. \\
 H\bigl( D_x V(t,x) \bigr) \leq 0.
\end{cases}
\end{equation}
Now, suppose $H(D_x V(t,x)) < 0$. Then $\hat H(p)=0$ and the optimal control is indeed $a=0$, with the uncontrolled HJB equal to zero:
\begin{equation}
 H\bigl( D_x V(t,x) \bigr) < 0 \quad \Longrightarrow \quad  \frac{\partial V(t,x)}{\partial t} + \hat f(t,x) + \LL^{a=0}V(t,x) = 0.
\end{equation}
The last equation can be written in a more compact form. We have the following \textbf{variational inequality}:
\begin{equation}\label{variational_inequality}
 \max \biggl\{ \frac{\partial V(t,x)}{\partial t} + \hat f(t,x) + \LL^{a=0}V(t,x) \,,\, H\bigl( D_x V(t,x) \bigr) \biggr \} =0 \quad \mbox{for} \quad (t,x)\in Q. 
\end{equation}

\subsection{General formulation}

Let us consider a state system governed by the following SDE:
\begin{align}\label{singular_SDE}
 dX_t =& \hat b(t,X_{t-}) dt + \hat \sigma (t,X_{t-}) dW_t \\ \nonumber
      &+ \int_{\R} \hat \gamma(t,X_{t-},z) \tilde N(dt,dz) + \kappa(t,X_{t-}) d\xi_t .
\end{align}
where $\kappa : [0,T] \times \R^n \to \R^{n \times m}$ is continuous. 
The m-dimensional process $\xi : [0,T] \times \Omega \to \R^m$ is a predictable, cádlág, bounded variation, 
non-decreasing process, with $\xi(0^-) = 0$. Let us denote with $\Pi$ the space of all controls $\xi$ and with $\Pi(t,x)$ the subset dependent on the current state $(t,x)$.   

For $\tau$ defined in (\ref{exit_time_def}), the objective function is:
\begin{equation}\label{singular_cost_functional}
 J^{\xi}(t,x) = \E_{t,x} \biggl[ \int_t^{\tau} f(s,X_s) ds + \int_t^{\tau} h(s,X_{s^-}) d\xi(s) + g(\tau, X_{\tau}) \biggr] . 
\end{equation}
where $f$, $h: [0,T] \times \R^n \to \R^{n \times m}$ and $g$ are continuous functions satisfying the quadratic growth condition. The value function is
\begin{equation}\label{controlled_VF}
 V(t,x) = \sup_{\Pi} J^{\xi}(t,x).
\end{equation}
Under certain assumptions, it can be proved that the value function (\ref{controlled_VF}) is a viscosity solution of a variational inequality 
with a form as in (\ref{variational_inequality}).

Let us assume that the value function satisfies the DPP. A straightforward modification of (\ref{DPP1}) and (\ref{DPP2}) yields the expressions: 
\begin{Theorem} \textbf{DPP for singular control}:
 \begin{itemize}
  \item For all $\xi \in \Pi(t,x)$ and all stopping time $\theta \in \mathcal{T}_{t,\tau}$:
  \begin{equation}\label{DPP11}
   V(t,x) \geq \E_{t,x} \biggl[ \int_t^{\theta} f(s,X_s) ds + \int_t^{\theta} h(s,X_{s^-}) d\xi(s) + V(\theta, X_{\theta}) \biggr]. 
  \end{equation}
  \item For all $\epsilon > 0$, there exists $\xi \in \Pi(t,x)$ such that $\forall \theta \in \mathcal{T}_{t,\tau}$: 
  \begin{equation}\label{DPP22}
      V(t,x) \leq \E_{t,x} \biggl[ \int_t^{\theta} f(s,X_s) ds + \int_t^{\tau} h(s,X_{s^-}) d\xi(s) + V(\theta, X_{\theta}) \biggr] + \epsilon. 
  \end{equation}
 \end{itemize}
\end{Theorem}


\noindent
In the previous section we derived the variational inequality using an heuristic argument. In this section, the assumptions (\ref{unbounded_set}) and (\ref{linear_control})
are replaced by considering directly the integral of $\alpha_t$ as our main control, and by enlarging the class of controls such that it can contain non absolute continuous processes
as well (we refer to Chapter 8.3 of \cite{FlemingSoner} for more details). 


\section{Viscosity solutions theory}


This section is dedicated to the definition of viscosity solutions.
In the literature there are different definitions of viscosity solution depending on the context. 
For instance, the theory presented in \cite{Pham} considers only the PDE case, while in \cite{Cont}
only linear PIDEs are considered. 
Other important references are \cite{FlemingSoner}, \cite{Ph98} and \cite{BaIm08} among others.
In the general discontinuous viscosity solutions approach, there is no
need to prove a priori the continuity of the value function $V$. The continuity will actually follow from
a strong comparison principle. 

\noindent
Here we present the definition for continuous viscosity solutions introduced in \cite{Kab16}, which is suitable for the problem proposed in this thesis (Chapter \ref{Chapter5}).


% Recall the definition of semi-continuous function:
% \begin{Definition}
%  Suppose $\mathcal{X}$ is a topological space and $x_0$ is a point in $\mathcal{X}$.
%  The function $f: \mathcal{X} \to \R$ is \textbf{upper semi-continuous} (USC) in $x_0$ if for every $\epsilon > 0$ exists a neighborhood of $x_0$, $U_{x_0}$, such that
%  $$ f(x) \leq f(x_0) + \epsilon \quad \quad \forall x \in U_{x_0}. $$
% \end{Definition}
% \begin{Definition}
%  Under the same assumptions,
%  the function $f: \mathcal{X} \to \R$ is \textbf{lower semi-continuous} (LSC) in $x_0$ if for every $\epsilon > 0$ exists a neighborhood of $x_0$, $U_{x_0}$, such that
%  $$ f(x) \geq f(x_0) - \epsilon \quad \quad \forall x \in U_{x_0}. $$
% \end{Definition}



\subsection{Definition viscosity solution}


Let us consider a general parabolic problem:
\begin{equation}\label{parabolic_PIDE}
\begin{cases}
F(t,x,u,D_t u,D_x u,D_{xx}u,\I(t,x,u)) = 0 \quad \mbox{ for } \quad (t,x) \in Q \\
u(t,x) = g(t,x) \quad \mbox{ for } \quad (t,x) \not \in Q
\end{cases}
\end{equation}
where $g \in C^0 \bigcap \mathcal{C}_2([0,T] \times (\R^n \backslash \OO))$ is a given function and 
$F$ is a continuous function that satisfies the following elliptic/parabolic local and non local conditions. 
For all $t\in [0,T); \; x \in \OO; \; r,\hat r \in \R; \; q,\hat q \in \R; \; 
p \in \R^n; \; M,\hat M \in \SI^n; \; \I, \hat \I \in \R$:
\begin{itemize}
 \item $r \leq \hat r \; \Longrightarrow  \; F(t,x,r,q,p,M,\I) \leq F(t,x,\hat r,q,p,M,\I) $
 \item $q \leq \hat q \; \Longrightarrow  \; F(t,x,r,q,p,M,\I) \geq F(t,x,r,\hat q,p,M,\I) $
 \item $M \leq \hat M \; \Longrightarrow  \; F(t,x,r,q,p,M,\I) \geq F(t,x,r,q,p,\hat M,\I) $
 \item $\I \leq \hat \I \; \Longrightarrow  \; F(t,x,r,q,p,M,\I) \geq F(t,x,r,q,p,M,\hat \I). $
\end{itemize}
where the matrix ordering is intended with this meaning:
$$ \hat M \geq M \Leftrightarrow \hat M - M \mbox{ is positive semi-definite.} $$

Having in mind our specific problem, we will assume that the viscosity subsolution and supersolution are continuous on $[0,T]\times \R^n$.
We can now introduce the definitions:
\begin{Definition}
 A continuous function $u$ is a \textbf{viscosity subsolution} of (\ref{parabolic_PIDE})
 if for any $(\bar t, \bar x) \in [0,T]\times \R^n$ and any test function $ \phi \in C^{1,2}([0,T] \times \R^n) \bigcap \mathcal{C}_2([0,T] \times \R^n)$ 
 such that $u-\phi$ has a global maximum at $(\bar t,\bar x)$ the following is satisfied:
\begin{align}\label{subsolution}
 & F\biggl( \bar t,\bar x,u(\bar t,\bar x),D_t \phi(\bar t,\bar x), D_x \phi(\bar t,\bar x),D_{xx}\phi(\bar t,\bar x),\I(\bar t, \bar x,\phi(\bar t,\bar x))\biggr) \leq 0 \\ \nonumber
 &  \quad \quad \mbox{ for } \quad (\bar t,\bar x) \in Q.
\end{align}
\end{Definition}

\noindent
In the same way we define:
\begin{Definition}
 A continuous function $u$ is a \textbf{viscosity supersolution} of (\ref{parabolic_PIDE})
 if for any $(\bar t, \bar x) \in [0,T]\times \R^n$ and any test functions $ \phi \in C^{1,2}([0,T] \times \R^n) \bigcap \mathcal{C}_2([0,T] \times \R^n)$ 
 such that $u-\phi$ has a global minimum at $(\bar t,\bar x)$ the following is satisfied:
\begin{align}\label{supersolution}
 & F\biggl( \bar t,\bar x,u(\bar t,\bar x),D_t \phi(\bar t,\bar x), D_x \phi(\bar t,\bar x),D_{xx}\phi(\bar t,\bar x),\I(\bar t, \bar x,\phi(\bar t,\bar x))\biggr) \geq 0 \\ \nonumber
 &  \quad \quad \mbox{ for } \quad (\bar t,\bar x) \in Q.
\end{align}
\end{Definition}

\noindent
When a function is both a viscosity subsolution and supersolution, it is called a \textbf{viscosity solution}.

% \begin{align}\label{subsolution}
%  & F\biggl( \bar t,\bar x,u(\bar t,\bar x),D_t \phi(\bar t,\bar x), D_x \phi(\bar t,\bar x),D_{xx}\phi(\bar t,\bar x),\I(\bar t, \bar x,\phi(\bar t,\bar x))\biggr) \leq 0 \\ \nonumber
%  &  \quad \quad \mbox{ for } \quad (\bar t,\bar x) \in Q, \\ \nonumber
%  & \min \biggl\{ F \biggl( \bar t,\bar x,u(\bar t,\bar x), D_t \phi(\bar t,\bar x), D_x \phi(\bar t,\bar x), 
%  D_{xx}\phi(\bar t,\bar x),\I(\bar t, \bar x,\phi(\bar t,\bar x)) \biggr) ,\\ \nonumber 
%  & \quad \quad u(\bar t,\bar x) - g(\bar t,\bar x) \biggr\} \leq 0  \quad  \mbox{ for } \quad (\bar t,\bar x) \in \partial Q, \\ \nonumber
%  & u(\bar t,\bar x) - g(\bar t,\bar x) \leq 0  \quad  \mbox{ for } \quad (\bar t,\bar x) \not \in \bar Q.
% \end{align}
% 
% \begin{align}\label{supersolution}
%  & F\biggl( \bar t,\bar x,u(\bar t,\bar x),D_t \phi(\bar t,\bar x), D_x \phi(\bar t,\bar x),D_{xx}\phi(\bar t,\bar x),\I(\bar t, \bar x,\phi(\bar t,\bar x))\biggr) \geq 0 \\ \nonumber
%  &  \quad \quad \mbox{ for } \quad (\bar t,\bar x) \in Q, \\ \nonumber
%  & \max \biggl\{ F \biggl( \bar t,\bar x,u(\bar t,\bar x), D_t \phi(\bar t,\bar x), D_x \phi(\bar t,\bar x), 
%  D_{xx}\phi(\bar t,\bar x),\I(\bar t, \bar x,\phi(\bar t,\bar x)) \biggr) ,\\ \nonumber 
%  & \quad \quad u(\bar t,\bar x) - g(\bar t,\bar x) \biggr\} \geq 0  \quad  \mbox{ for } \quad (\bar t,\bar x) \in \partial Q, \\ \nonumber
%  & u(\bar t,\bar x) - g(\bar t,\bar x) \geq 0  \quad  \mbox{ for } \quad (\bar t,\bar x) \not \in \bar Q.
% \end{align}
