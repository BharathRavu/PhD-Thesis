
\chapter*{Resumo alargado}    


Na área da Matemática Financeira, um dos problemas fundamentais é a determinação do “preço justo” de uma opção. 
Este problema não é trivial e para o resolver é necessário considerar um modelo estocástico adequado para a dinâmica do ativo financeiro subjacente 
e também algumas hipóteses sobre o funcionamento do mercado. 

Uma solução para o problema de avaliação de opções foi apresentada pela primeira vez no célebre artigo de  Fischer Black e Myron Scholes (\cite{BS73}). 
No modelo de Black-Scholes standard assume-se um mercado sem fricções, onde a taxa de juro é constante e o processo de preço do ativo subjacente é um movimento Browniano geométrico. 
Este modelo é, ainda hoje, um dos modelos mais usados para avaliar opções e definir estratégias de cobertura de risco. 
As razões para a popularidade deste modelo prendem-se sobretudo com o facto dele fornecer fórmulas fechadas para o cálculo do preço das opções mais simples 
e permitir estratégias de cobertura de risco facilmente implementáveis. 

Apesar do modelo de Black-Scholes ser bastante popular, a aplicação do modelo pressupõe hipóteses que não refletem as verdadeiras características observadas empiricamente no mercado. 
Em particular, a hipótese de que o processo de preço do ativo subjacente é um movimento Browniano geométrico, o que implica uma distribuição dos retornos logarítmicos Gaussiana, 
é frequentemente rejeitada através de análises estatísticas rigorosas dos dados empíricos. 
As distribuições empíricas dos retornos logarítmicos são habitualmente caracterizadas por uma maior massa na vizinhança do centro da distribuição e por caudas pesadas 
(quando comparadas com a distribuição normal). 

De forma a ultrapassar a inconsistência entre o movimento Browniano geométrico e os dados empíricos, vários autores consideraram processos estocásticos alternativos 
para descrever a dinâmica dos preços do ativo subjacente. 
Em particular, os processos de Lévy tornaram-se relativamente populares nas últimas duas décadas e são bons candidatos para reproduzir de forma mais precisa as distribuições empíricas 
dos retornos logarítmicos. 
Estes processos têm boas propriedades analíticas e incluem processos associados a distribuições com caudas pesadas. 
Os processos de Lévy são, essencialmente, processos estocásticos com incrementos independentes e estacionários, que satisfazem também a propriedade de continuidade estocástica. 
Nesta tese consideraremos em geral processos de Lévy para modelar o processo de preço do ativo subjacente e, em particular, consideraremos como principais exemplos 
o movimento Browniano, o modelo de difusão com saltos de Merton e o processo “Variance Gamma”, que é um processo de atividade infinita.  

A hipótese de mercado sem fricções, que é assumida no modelo de Black-Scholes standard, é também francamente irrealista, 
porque os mercados financeiros são habitualmente caracterizados por diferentes tipos de taxas cobradas aos investidores sempre que estes realizam transações. 
Em particular, as transações realizadas em tempo contínuo, tal como prescritas pelo modelo de Black-Scholes, não são praticáveis na realidade. 
Além disso, a presença de custos ou taxas de transação impede a replicação perfeita do “payoff” da opção. 

Nesta tese apresentamos um novo modelo para avaliação de opções na presença de custos de transação, 
considerando que o processo de preço do ativo subjacente é um processo de Lévy exponencial. 
O modelo proposto é a generalização do modelo introduzido inicialmente por \cite{HoNe89} e mais tarde formalizado por 
\cite{DaPaZa93}, onde apenas o termo de difusão é considerado na dinâmica do preço. 
No modelo estudado nesta tese introduzimos também a possibilidade de incumprimento (ou “default”) para a carteira do investidor. 
No nosso modelo, o preço de uma opção é definido usando o conceito de “preço de indiferença”, considerando funções de utilidade e a teoria de otimização de carteiras em tempo contínuo. 
O preço de uma opção é o valor que faz com que a escolha de um investidor seja indiferente (em termos da utilidade esperada da riqueza final) 
entre as possibilidades de ter uma carteira com uma opção e ter uma carteira com a opção substituída por um prémio de risco fixo. 
Na presença de custos de transação, a carteira de replicação perfeita não pode ser implementada. 
Como consequência, a estratégia de cobertura de risco deixa de ser uma estratégia sem risco.  
A estratégia de cobertura de risco ótima consiste em manter o valor da carteira dentro de uma região designada por “região de não transação”.  
Considerando a abordagem descrita em \cite{Kab16}, formulamos o nosso problema de avaliação de opções com custos de transação como um problema de controlo singular ótimo 
e deduzimos a equação de Hamilton-Jacobi-Bellman (HJB) para este problema. 
Esta equação é uma equação integro-diferencial parcial (EIDP), que toma a forma de uma desigualdade variacional. 
Assumimos que a função valor do problema é contínua e satisfaz o princípio de programação dinâmica. 
Sob esta hipótese, provamos que a função valor é uma solução da equação HJB. 
Esta solução deve ser interpretada como uma solução de viscosidade para o problema. 
O problema de otimização original é complexo e obter uma aproximação numérica para a solução é, do ponto de vista computacional, um processo bastante intenso. 
Para simplificar o problema considerámos o caso particular de um investidor com grandes recursos de crédito, de forma a ignorar a possibilidade de “default”. 
Sob esta hipótese, e devido às propriedades da função utilidade exponencial, é possível reduzir o número de variáveis do problema e obter uma versão tridimensional do mesmo 
e da equação HJB associada. 

Um algoritmo é proposto para obter soluções numéricas para o problema original e para o problema simplificado. 
No entanto, foi dada maior enfâse ao problema simplificado, uma vez que neste caso a complexidade computacional é menor e foi possível realizar mais estudos numéricos. 
A abordagem numérica baseia-se no método de aproximação por cadeias de Markov proposto em \cite{Kushner}. 
A dinâmica da carteira é aproximada por uma cadeia de Markov em tempo discreto. 
Uma forma de construir a cadeia de Markov é através da discretização do gerador infinitesimal do processo, usando um método de diferenças finitas explícito. 
Este tipo de construção é imediata para processos de atividade finita como o processo de difusão com saltos de Merton, mas para processos de atividade infinita, 
como o processo “Variance-Gamma”, não é claro como obter as probabilidades de transição por este método. 
Um procedimento alternativo habitual é aproximar a componente de martingala dos “saltos pequenos” do processo de atividade infinita por um movimento Browniano com a mesma variância, 
de forma a remover a singularidade da medida de Lévy na vizinhança da origem. 
Relativamente ao método numérico, este consiste essencialmente em criar um algoritmo de programação dinâmica recursivo, 
de forma a calcular a função valor no instante $t$ a partir do seu valor no instante $t+\Delta t$. 
Em \cite{Kushner}, os autores provaram que a função valor obtida através de um algoritmo de programação dinâmica discreta deste tipo converge para a função valor do problema 
original usando um argumento de convergência em probabilidade. 
De forma a provar a convergência do nosso esquema numérico, usamos outra abordagem, baseada no método de prova de convergência para a solução de viscosidade introduzido 
em \cite{BaSo91}. 
Provamos que o esquema é monótono, estável e consistente. Provamos também que a solução obtida pelo esquema numérico converge para a solução de viscosidade da equação HJB. 

Exemplos numéricos são apresentados para os casos particulares em que a dinâmica do preço do ativo subjacente é dada por: 
(i) Movimento Browniano; (ii) Processo de difusão com saltos de Merton; (iii) Processo “Variance-Gamma”.  
Comparamos os resultados obtidos pelo nosso modelo com os preços obtidos ao resolver a EIDP deduzida a partir da teoria de avaliação pelo método de martingala standard. 
Para pequena aversão ao risco e para custos de transação nulos, o nosso modelo replica os preços obtidos a partir da EIDP com boa precisão.  
As EIDP foram resolvidas através do esquema de diferenças finitas do tipo implícito-explícito proposto em \cite{CoVo05b}. 

Nesta tese discutimos também o método multinomial proposto em \cite{YaPr01}, que é um método alternativo que permite obter as probabilidades de transição da cadeia de Markov aproximadas. 
Aplicamos este método ao processo “Variance Gamma”, o qual é previamente aproximado por um processo de saltos com os mesmos quatro primeiros momentos e depois é discretizado 
no tempo e no espaço. 
Comparamos os resultados obtidos pelo método multinomial com os resultados obtidos a partir da solução da EIDP para o caso de opções Europeias e Americanas standard. 
Aplicamos também o método multinomial ao modelo com custos de transação proposto nesta tese. 
Neste método não há necessidade de discretizar o gerador infinitesimal para a construção da cadeia de Markov. 
Uma vez que o número de ramos no método multinomial proposto é mantido fixo e igual a cinco, a complexidade computacional é mais pequena. 

 \vspace{1em}
 \noindent \textbf{Palavras-chave:} preços de opções, processos de Lévy, custos de transação, aproximação da cadeia de Markov, controlo singular.