

\chapter{Introduction}\label{Introd}
%\blindtext
\minitoc% Creating an actual minitoc


\vspace{5em}

The problem of pricing a European call option was first solved mathematically in the paper of \cite{BS73}. 
Even if it is quite evident that this model is too simplistic to represent the real features of the market, it is 
still nowadays one of the most used model to price and hedge options.
The reason for its success is that it gives a closed form solution for the option price, and that the hedging strategy is easily 
implementable.
The Black-Scholes model considers a \emph{complete} market, i.e. a market where it is possible to create a portfolio containing cash 
and shares of the underlying stocks, such that following a particular trading strategy it is always possible to replicate
the payoff of the option. In this framework, this particular portfolio is called \emph{replicating portfolio} and
the trading strategy to hedge the option is called \emph{delta hedge}.
However, this model does not consider many features that characterize the real market. 

In the Black-Scholes model 
the stock price follows a geometric Brownian motion. This is equivalent to assume that the log-returns are 
normally distributed. 
However, a rigorous statistical analysis of financial data
reveals that the normality assumption is not a very good approximation of
reality (see \cite{Cont01}). Indeed, it is easy to see that empirical log-return distributions have
more mass around the origin and along the tails (\emph{heavy tails}).
This means that normal distribution underestimates the probability of large positive and negative
log-returns, and considers them just as rare events. In the real market instead,
log-returns manifest frequently high peaks, that come more and more evident
when looking at short time scales. The log-returns peaks correspond to sudden
large changes in the price, which are called \emph{jumps}. 
There is a huge literature of option pricing models that consider an underlying process with a discontinuous path.
Most of these models consider the log-prices dynamics following a \emph{Lévy process}. 
 

A second issue of the Black-Scholes model is that it does not consider the presence of market frictions i.e.
bid/ask spread, transaction fees and budget constraints.
The securities in the market are traded with a bid-ask spread, and this means that there are two prices for the
same security. But the Black-Scholes formula just gives one price.
Moreover, the replicating portfolio cannot be perfectly implemented,
since the delta-hedging strategy involves continuous time trading. 
This is impractical because the presence of transaction costs makes it infinitely costly.
Another kind of market friction that needs to be considered are the budget constraints. 
A bound in the budget or a restriction in the possibility of 
selling short, clearly restricts the set of possible trading strategies.

Many authors attempted to include the presence of proportional transaction costs in option pricing models.
In \cite{Le85}, in order to avoid continuous trading, the author specifies 
a finite number of trading dates. He obtains a Black-Scholes-like
nonlinear partial differential equation (PDE) with an adjusted volatility term, that takes into account the transaction costs. 
However, trading at fixed dates is not optimal, and the option price goes to infinity as the number of dates grows.
Further work in this direction has been done by \cite{BoVo92}, which consider a multi-period binomial model (see \cite{CRR79})
with transaction costs. Here again, the cost of the replicating portfolio depends on the number of time periods. 
Recent developments in this direction are for instance \cite{Mocio07}, \cite{FlMaSe14} and \cite{Sengu14} 
who consider different features of the market such as jumps, stochastic volatility and stochastic interest rate respectively.  

A different approach has been introduced by \cite{HoNe89}. The authors use an alternative definition of the option price
called \emph{indifference price}, based on the concepts of \emph{expected utility} and \emph{certainty equivalent}.  
An overview of these concepts applied to several incomplete market models can be found in \cite{Carmona}.
As long as the perfect replicating portfolio is no longer implementable in presence of transaction costs, the 
hedging strategy cannot be anymore riskless. 
The model has to take into account the risk profile of the writer/buyer to describe his trading preferences.
\cite{HoNe89} define the option price as the value that makes an investor indifferent between holding a portfolio with an option
and without, in terms of expected utility of the final wealth.
They show that it is impossible to hedge perfectly the option. The optimal strategy is to keep the portfolio's values within
a band called \emph{no transaction region}. Using numerical experiments, they verify that this strategy outperforms the one 
proposed in \cite{Le85}.
This approach has been further developed in \cite{DaPaZa93}, where the problem is formulated rigorously as a singular 
stochastic optimal control problem. The authors prove that the value functions of the two optimization problems
can be interpreted as the solutions of the associated Hamilton-Jacobi-Bellman (HJB) equation in the viscosity sense. 
They prove also that the numerical scheme, based on the \emph{Markov chain approximation}, converges to the viscosity solution.
Numerical methods for this model are presented in \cite{DaPa94}, \cite{ClHo97} and \cite{Mon03}, \cite{Mon04}.
In \cite{WhWi97} and \cite{BaSo98} the problem is simplified by using asymptotic analysis methods for small levels of 
transaction costs. The authors, starting from the general HJB variational inequality, derive a simpler non-linear PDE for the option price. 
Further developments are presented in the thesis work of \cite{Damgaard}, where the author 
studies the robustness of the model from a theoretical and numerical point of view. 
He found that under particular conditions the model is quite robust with respect to the choice of the utility function. 

In this thesis the main goal is to develop and analyze a model for pricing options when the market is incomplete due to the presence of jumps and transaction costs. 
The topics of the thesis are based on the two papers 
\cite{Canta2}, \cite{Canta} and the contributed chapter \emph{Indifference pricing in a market with transaction costs and jumps} by N. Cantarutti, J. Guerra, M. Guerra and 
M.R. Grossinho, published in the book of \cite{Matthias}.\\
Portfolio models with transaction costs and Lévy processes have already been introduced in the financial literature, see for instance \cite{OkSu01}, \cite{BKR01} and \cite{Kab16} 
but these models have never been used to price options.  
We analyze its theoretical properties, such as the existence of a viscosity solution \todo{decide if put the proofs of viscosity solutions} 
of the nonlinear \todo{This part can be modified later}
Partial Integro-Differential Equation (PIDE) associated to the model, and develop numerical methods to solve the problem under different assumptions, i.e. ignoring the 
possibility of default and not.
We present numerical results that are new in the literature and compare them with numerical values obtained from existing reference models. We also prove that 
the proposed numerical scheme is monotone, stable and consistent, and its solution converges to the viscosity solution of the continuous problem.
\todo{I should add here the last results on computational time and rate of convergence.}

\todo{In the following lines the contents of the chapters can be modified later. Check it in the end.}
In Chapter \ref{Chapter1}, we introduce the general theory for Lévy processes, with a deeper focus on the particular Lévy processes that are commonly used in financial modeling:  
the \emph{Merton} and \emph{Variance Gamma} processes.   
In Chapter \ref{Chapter2} we make a small summary of the main assumptions and theorems of the \emph{No arbitrage theory} for derivative pricing. After that we present 
the most common numerical finite differences methods used to solve the option PIDE. 
The Chapter \ref{Chapter3} is a digression based on the paper \cite{Canta2} on the application of the multinomial method to solve option pricing problems under the Variance Gamma process. 
In Chapter \ref{Chapter4} we present the general optimal control theory for jump processes and the definition of viscosity solutions. 
Under this framework, in Chapter \ref{Chapter5}, we will develop the model for option pricing in presence of proportional transaction costs. This model is a singular stochastic
control problem, which is a generalization of the \cite{DaPaZa93} model. We derive the general HJB equation and prove that the value function of the optimization problem 
can be interpreted as its viscosity solution.  
In Chapter \ref{Chapter6} we solve the problem numerically. 
First, we consider the simplified problem where the number of variables is reduced by one thanks to the assumption of no default and the property of the exponential utility. 
The numerical results are also presented in \cite{Canta} and \cite{Matthias}.\\
Then we solve the general problem with four variables, considering the possibility of default. In the end we also show that the moment matching method developed for 
the Variance Gamma process, can be used to solve these kind of problems with good performance. 

